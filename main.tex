\documentclass[12pt]{amsart}
\usepackage[utf8]{inputenc}
\usepackage{ae,aecompl,aeguill}	% pour utiliser << et >>
\usepackage[english]{babel}
\usepackage{times}
\usepackage[babel=true,kerning=true]{microtype} 
\usepackage{enumerate}

\usepackage{amsthm}
\usepackage{bbm}

\newtheorem{proposition}{Proposition}[section]
\newtheorem{remark}{Remark}[section]
\newtheorem{example}{Example}[section]

\renewcommand{\sl}{\operatorname{sl}}
\newcommand{\SL}{\operatorname{SL}}
\newcommand{\p}{\mathbbm{1}} % p for pointed set
\renewcommand{\H}{\mathcal{H}}

\begin{document}

\begin{abstract}
  Let $M$ be a monoid of transformations of a set $\Omega$, defined by
  generators. The basic operation in algorithms such as Froidure-Pin
  to compute the structure of $M$ is the composition of an element of
  $M$ and a generator, of complexity $O(|\Omega|)$. When the set $\Omega$
  is large the information contained in a transformation can be very
  redundant. For example, for the $0$-Hecke monoid $H_n(0)$, $\Omega$ is
  the symmetric group $S_n$ of size $n!$, whereas each transformation
  is uniquely determined by the image of the identity in $S_n$.

  It therefore would be desirable to perform partial computation of
  the transformations, typically by only recording the image of some
  distinguished elements of $\Omega$. Now what if information is lost
  in the process?

  In this paper, we prove that any loss of information will
  materialize in a failure of associativity which can be detected
  during the execution of a slightly modified Froidure-Pin algorithm.
\end{abstract}

Our conjecture: if injectivity is not satisfied, then it's going to
show up in the associativity test of Froidure Pin.



Consider a set $\Omega$ with a distinguished element $\p$. We denote
by $\alpha.f$ the image of a transformation $f$ on an element $\alpha$
of $\Omega$. Take a family of transformations
$(\pi_i: \Omega \to \Omega)_{i\in I}$, where $I$ is some finite
alphabet. We let the free monoid $I^*$ acts naturally on the right on
$\Omega$ by
\begin{displaymath}
  \alpha.(i_1i_2\cdots i_k) = \alpha.\pi_{i_1}.\pi_{i_2}\ldots
  \pi_{i_k}, \text{ for } i_1,i_2,\ldots,i_k\in I.
\end{displaymath}
For a word $u$, we denote by $\pi_u$ the corresponding transformation.
Let $M$ be the image of $\pi: u\mapsto \pi_u$ or, equivalently, the
appropriate quotient monoid of $I^*$.

Our wish is to apply the Froidure Pin algorithm to $M$ while
identifying the right Cayley graph of $M$ and the graph of its action
on $\p.M$ through the map $m \mapsto \p.m$.

We consider the shortlex section $\sl$ of this map: for all $m$ in
$M$, we let $\sl(m)$ be the least shortlex word such that
$\p.m = \p.\sl(m)$. By extension, for any word $w$ we set
$\sl(w)=\sl(\pi_m)$. We define:
\begin{displaymath}
  \SL = \{ \sl(m),  m \in M \}\,.
\end{displaymath}

\begin{example}
  Let $\Omega = \{1,2,3,4,5\}$, $\p =1$, and consider the monoid $M$
  generated by the permutation $(1,2),(3,4,5)$ as single
  transformation. If we apply the Froidure Pin algorithm by just
  looking at the orbit of $\p$, we blind ourselves from what's
  happening elsewhere in $\Omega$, and just find the order $2$ proper
  quotient of $M$ induced by its action on $\{1,2\}$. This is
  equivalent to applying the Froidure Pin algorithm to the
  transformation monoid generated by $(1,2)$, and no contradiction can
  be found.

  Therefore, we need to assume that the action of $M$ on $\p.M$ is
  faithful.

  To handle the above example, we may replace $\p$ by the pair
  $(1,3)$, and let $M$ act diagonally on pairs. Or take for $\p$ some
  larger base. The extreme is to take $\p = (1,2,3,4,5)$, and let $M$
  act on 5-tuples; this amounts to represent elements of $M$ by
  transformations.
\end{example}

\begin{proposition}
  Take $M$, $\p$, $\sl$ as above. Assume that the action of $M$ on
  $\p.M$ is faithful. Then the following are equivalent:
  \begin{enumerate}[(a)]
  \item $m \mapsto \p.m$ is injective;
  \item If $f,g\in M$, $i\in I$ and $\p.f = \p.g$ then
    $(\p.i).f = (\p.i).g$;
  \item For any $u\in I^*$ and $i\in I$, $(\p.i).u = (\p.i).\sl(u)$; %;  same for all f
  \item For any $u\in I^*$ and $i,j\in I$, $(\p.i).\sl(u)j = (\p.i).\sl(uj)$.
  \end{enumerate}
\end{proposition}

\begin{remark}[Application of (d) to Froidure Pin]

  Let's consider the step where we have some already computed vertex
  $\p.u$ where $u$ is in $\SL$, and we compute $\p.u.j$ to see where
  it goes under the action of $j$. Assume that there is a collision:
  $\sl(uj) < uj$ in shortlex order.

  By Froidure Pin, we have already used associativity to compute the
  left actions on $u$ and $\sl(uj)$, namely $(\p.i).u$ and
  $(\p.i).\sl(uj)$ Usually, we don't bother computing the left action
  on $\p.uj$, since it's not a new element.

  But let's do it; namely calculate and compare $(\p.i).uj$ and
  $(\p.i).\sl(uj)$. If they are not equal, then stop: the map $\p.$ is
  not injective.
\end{remark}

\begin{proof}
(a) => (b): obvious.

(b) => (a):

We will use the following induction hypothesis $H_l$:

    for all u, f,g with |u| \leq l,    \p.f = \p.g  implies (\p.u).f = (\p.u).g

Note that \H_0 is trivial, \H_1 is (b) and (a) is \H_\infty.

Let l > 1 and assume that \H_l holds. Let us prove that \H_{l+1} holds.

Take v = ui of length l+1, and f and g such that \p.f=\p.g. By \H_1,

     \p . (\pi_i f) = (\p.i). f = \p.(\pi_i g) = \p . (\pi_i g)

Applying \H_l to \pi_i f and \pi_i g, we obtain:

       (\p.ui) . f  =  (\p.u) . (\pi_i f) = (\p.u) . (\pi_i g)
                    =  (\p.ui) . f,

as desired.

(b) => (c): \TODO{write the proof; very similar to below}

(c) => (b): Let's assume that $\p.f = \p.g$; then $\sl(f) = \sl(g)$,
and using (c),
\begin{displaymath}
 (\p.i).f = (\p.i).sl(f) = (\p.i).sl(g) = (\p.i).g\,.
\end{displaymath}

(d) => (c):

Let u = j_1 ... j_k. Then,

    (\p.i).u = (\p.i).sl()j_1j_2 ... j_k
            = (\p.i).sl(j_1)j_2 ... j_k
            = (\p.i).sl(j_1j_2) ... j_k
            = (\p.i).sl(j_1j_2 ... j_k)
            = (\p.i).sl(u)

\end{proof}
\end{document}
